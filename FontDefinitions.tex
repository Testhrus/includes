% Schriftdefinitionen
\ifxetex
\usepackage{xunicode}
\fi

\ifpdftex
\else
    \usepackage{fontspec}
\fi

\providecommand{\fonttype}{libertinus}

\ifthenelse{\equal{\fonttype}{basel}}
{
    \setmainfont[
        Path                = \mainpath/includes/fonts/,
        Extension           = .ttf,
        UprightFont         = *_Normal,
        BoldFont            = *_Bold,
        ItalicFont          = *_Italic,
        BoldItalicFont      = *_BoldItalic,
        %
        SmallCapsFeatures   = {Letters=SmallCaps},
        %Mapping             = tex-text,
        Ligatures           = TeX
    ]{BaselBook}

    \newcommand\FontName{Basel Book\xspace}
    \typeout{ **** Use BaselBook ***}
}{}

\ifthenelse{\equal{\fonttype}{libertine}}
{
    \setmainfont[
        Path                = \mainpath/includes/fonts/LinuxLibertine/,
        Extension           = .ttf,
        UprightFont         = *_R_G,
        BoldFont            = *_RB_G,
        ItalicFont          = *_RI_G,
        BoldItalicFont      = *_RBI_G,
        %
        SmallCapsFeatures   = {Letters=SmallCaps},
        %Mapping             = tex-text,  not for luatex
        Ligatures           = TeX
    ]{LinLibertine}

    \newcommand\FontName{Linux Libertine\xspace}
    \typeout{ **** Use LinuxLibertine ***}
}{}

\ifthenelse{\equal{\fonttype}{libertinus}}
{
    \usepackage{libertinus-otf}
    \setmainfont{Libertinus Serif}
    \setsansfont{Libertinus Sans}
    %\setmathfont{Libertinus Math}
    \newcommand\FontName{Libertinus\xspace}
    \typeout{ **** Use Libertinus ***}
}{}

\defaultfontfeatures{Scale=MatchLowercase}
