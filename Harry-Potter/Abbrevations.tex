% Harry Potter Abkürzungen

% Links und Rechts vom Rand einrücken und kein Blocksatz
\newenvironment{brief}     {\vspace{\baselineskip}\begin{addmargin}[1.5em]{1.5em}\begin{FlushLeft}\itshape}
                                {\end{FlushLeft}\end{addmargin}\skipspace\vspace{\baselineskip}}
\newenvironment{buch}      {\vspace{\baselineskip}\begin{addmargin}[1.5em]{1.5em}\begin{FlushLeft}\itshape}
                                {\end{FlushLeft}\end{addmargin}\skipspace\vspace{\baselineskip}}
\newenvironment{lied}      {\vspace{\baselineskip}\begin{addmargin}[1.5em]{1.5em}\begin{FlushLeft}\itshape}
                                {\end{FlushLeft}\end{addmargin}\skipspace\vspace{\baselineskip}}

\newenvironment{traum}     {\vspace{\baselineskip}\itshape}{\skipspace\vspace{\baselineskip}}
\newenvironment{rueckblick}{\vspace{\baselineskip}\itshape}{\skipspace\vspace{\baselineskip}}

\newenvironment{rolle}{\sfshape}{}



\newcommand{\ps}[1]{\skipspace #1\\}
\newcommand{\addressingspace}{\\ \\} % Achtung! Alt+255
\newcommand{\signumspace}{\\ \\} % Achtung! Alt+255

\newcommand{\gleis}{9~\nicefrac{3}{4}\xspace}
\newcommand{\neundreiviertel}{9~\nicefrac{3}{4}\xspace}

\newcommand{\kapitelvorwort}      [1]{\skipspace\begin{center}\footnotesize\textit{#1}\end{center}\skipspace}

\newcommand{\note}                [1]{\textit{#1}}
\newcommand{\parsel}              [1]{\textit{>>#1<<\xspace}}
\newcommand{\koboldgack}          [1]{\textit{‛#1’\xspace}}
\newcommand{\meerisch}            [1]{\textit{<<#1>>\xspace}}
\newcommand{\mensch}[1]{} % do nothing here, because of another command for output in an other language. This here is for transformation in Html with own tool
\newcommand{\muenze}              [1]{‚#1{}‘}
\newcommand{\fach}                [1]{\textit{#1}}
\newcommand{\buchtitel}           [1]{\textit{#1}}
\newcommand{\liedinline}          [1]{\textit{#1}}
\newcommand{\spruch}              [1]{\textit{#1}}
\newcommand{\zauber}              [1]{\enquote{\textit{#1}}}
\newcommand{\zauberextase}        [1]{\enquote{\textit{\textsc{#1}}}}
\newcommand{\trank}               [1]{\textit{#1}}

\newcommand{\fab}{Florish \& Blotts\xspace}
\newcommand{\schnatz}{Schnatz\xspace}

\newcommand{\VgddK}{\fach{Verteidigung gegen die dunklen Künste}\xspace}
\newcommand{\DKuV}{\fach{Dunkle Künste und Verteidigung dagegen}\xspace}

\newcommand{\nachrichten}         [1]{\textit{#1}}
\newcommand{\zeitungname}         [1]{\textit{#1}}
\newcommand{\zeitung}             [1]{\skipspace\begin{quote}#1\end{quote}\skipspace}

%\newcommand{\nachrichtensprecher} [1]{\enquote{\textit{#1}}}
%\newcommand{\letter}              [1]{‚#1{}‘}
%\newcommand{\rolle}               [1]{\textsf{#1}}
%\newcommand{\mitte}               [1]{\begin{center}\textit{#1}\end{center}}
%\newcommand{\unterrichtsfach}     [1]{\textit{#1}}
%\newcommand{\headline}            [1]{\textsc{#1}}
