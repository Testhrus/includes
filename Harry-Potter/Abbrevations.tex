% Harry Potter Abkürzungen
%%                                Achtung: Nach \skipspace kommt Alt+255 (am Anfang und am Ende)
\providecommand{\doIndentAndFloppy}[1]{\skipspace \vspace*{-0.25ex}\begin{addmargin}[1.5em]{1.5em}\begin{FlushLeft}\textit{#1}\end{FlushLeft}\end{addmargin}\skipspace \vspace*{-0.2ex}}

\newenvironment{briefu}{\skipspace \vspace*{-0.25ex}\begin{addmargin}[1.5em]{1.5em}\begin{FlushLeft} \itshape}{\end{FlushLeft}\end{addmargin}\skipspace \vspace*{-0.2ex}}
\newenvironment{liedu}{\skipspace \vspace*{-0.25ex}\begin{addmargin}[1.5em]{1.5em}\begin{FlushLeft} \itshape}{\end{FlushLeft}\end{addmargin}\skipspace \vspace*{-0.2ex}}
\newenvironment{rueckblicku}{\skipspace \vspace*{-0.25ex}\itshape}{\skipspace \vspace*{-0.2ex}}

\newcommand{\ps}            [1] {\skipspace #1\\}
\newcommand{\addressingspace}   {\\ \\} % Achtung! Alt+255
\newcommand{\signumspace}       {\\ \\} % Achtung! Alt+255

\newcommand{\fahrendeRitter}{\textit{Fahrende Ritter}}
\newcommand{\fahrenderRitter}{\textit{Fahrender Ritter}}
\newcommand{\fahrendenRitter}{\textit{Fahrenden Ritter}}

\newcommand{\gleis}{9~\nicefrac{3}{4}\xspace}
\newcommand{\neundreiviertel}{9~\nicefrac{3}{4}\xspace}

\newcommand{\timeline}            [1] {\begin{center}\textit{#1}\end{center}\skipspace}

\newcommand{\kapitelvorwort}      [1] {\skipspace\begin{center}\footnotesize\textit{#1}\end{center}\skipspace}
\newcommand{\centerblock}         [1] {\skipspace\begin{center}\textsc{#1}\end{center}\skipspace}
\newcommand{\angriffmark}         [1] {\skipspace\begin{flushleft}\textit{#1}\end{flushleft}\skipspace}
\newcommand{\lied}                [1] {\skipspace\begin{flushleft}\textit{#1}\end{flushleft}\skipspace}

\newcommand{\widmung}             [1] {\doIndentAndFloppy{#1}}
\newcommand{\gedicht}             [1] {\doIndentAndFloppy{#1}}
\newcommand{\danke}               [1] {\doIndentAndFloppy{#1}}
\newcommand{\buchabschnitt}       [1] {\doIndentAndFloppy{#1}}
\newcommand{\fernseher}           [1] {\doIndentAndFloppy{#1}}
\newcommand{\anschlag}            [1] {\doIndentAndFloppy{#1}}
\newcommand{\note}                [1] {\doIndentAndFloppy{#1}}
\newcommand{\phiole}              [1] {\doIndentAndFloppy{#1}}
\newcommand{\brief}               [1] {\doIndentAndFloppy{#1}}

\newcommand{\zeitung}             [1] {\skipspace\begin{quote}#1\end{quote}\skipspace}
\newcommand{\zeitungname}         [1] {\textit{#1}}
\newcommand{\headline}            [1] {\textsc{#1}}
\newcommand{\phioleschrift}       [1] {\textit{#1}}
\newcommand{\benotung}            [1] {\textit{#1}}
\newcommand{\parsel}              [1] {\textit{>>#1<<\xspace}}
\newcommand{\meerisch}            [1] {\textit{<<#1>>\xspace}}
\newcommand{\einfluss}            [1] {\enquote{\textsc{#1}}}
\newcommand{\muenze}              [1] {‚#1{}‘}
\newcommand{\letter}              [1] {‚#1{}‘}

\newcommand{\introduction}        [1]{\skipspace\begin{center}\small #1\end{center}\skipspace}
\newcommand{\rolle}               [1]{\textsf{#1}}
\newcommand{\mitte}               [1]{\begin{center}\textit{#1}\end{center}}
\newcommand{\unterrichtsfach}     [1]{\textit{#1}}
\newcommand{\fach}                [1]{\textit{#1}}
\newcommand{\buch}                [1]{\textit{#1}}
\newcommand{\nachrichten}         [1]{\textit{#1}}
\newcommand{\nachrichtensprecher} [1]{\enquote{\textit{#1}}}

\newcommand{\liedinline}          [1]{\textit{#1}}
\newcommand{\spruch}              [1]{\textit{#1}}
\newcommand{\zauber}              [1]{\enquote{\textit{#1}}}
\newcommand{\zauberextase}        [1]{\enquote{\textit{\textsc{#1}}}}

\newcommand{\fab}{Florish \& Blotts\xspace}
\newcommand{\schnatz}{Schnatz\xspace}

\newcommand{\VgddK}{\unterrichtsfach{Verteidigung gegen die dunklen Künste}\xspace}
