\usepackage{verbatim}

\makeatletter
\newcounter{exercise}[chapter]

% Datei zum Schreiben oeffnen:
\newwrite\ex@file
\immediate\openout\ex@file=\jobname.kommentar

% Wird bei \begin{kommentar} ausgefuehrt:
\newcommand*{\kommentar}{\@bsphack
\let\do\@makeother\dospecials
\catcode`\^^M\active
\def\verbatim@processline{%
\immediate\write\ex@file{\the\verbatim@line}}%
\stepcounter{exercise}%
\immediate\write\ex@file{%
\string\begin{ex@kommentar}{\thechapter.\theexercise}}%
\verbatim@start}

% Wird bei \end{kommentar} ausgefuehrt:
\def\endkommentar{%
\immediate\write\ex@file{%
\string\end{ex@kommentar}}\@esphack}


% Umgebung definieren, wie die Kommentare formatiert werden sollen
	\newenvironment{ex@kommentar}[1]{\paragraph{#1}}{}

% Dieser Befehl listet die Kommentare auf
  \newcommand*{\includekommentare}{%
  \immediate\closeout\ex@file%
  \@ifundefined{chapter}%
  {\let\@tempa\section}%
  {\let\@tempa\chapter}
  \@tempa{Kommentare}
  \InputIfFileExists{\jobname.kommentar}{}{}}

\makeatother
