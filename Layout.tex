\usepackage{zahl2string}        % Kapitel Eins
\usepackage{fmtcount}           % Chapter One
\usepackage{scrlayer-scrpage}   % z.B. \pagestyle, \ohead, …

\providecommand{\papertype}{dina5}
\providecommand{\chapterword}{\iflanguage{german}{\Numstring{chapter}}{\iflanguage{ngerman}{\Numstring{chapter}}{\Numberstring{chapter}}}}

\KOMAoption{chapterprefix}{true}
\renewcommand*\raggedchapter{\centering}
\newcommand*\chaptertitleformat[1]{#1}
\setkomafont{chapterprefix}{\rmfamily\Large\normalshape\mdseries} %% Formatierung: - Kapitel Eins -
\setkomafont{chapter}{\rmfamily\Huge\itshape\bfseries} %% Formatierung: <Kapitelname>

\ifthenelse{\equal{\papertype}{dina5}}
{
    \renewcommand*{\chapterformat}{ \vspace*{-12.5mm} – \scshape \chaptername\ \chapterword – }
    \renewcommand*{\chapterheadendvskip}{\vspace*{14.7mm}}
}{}

\ifthenelse{\equal{\papertype}{science}}
{
    \renewcommand*{\chapterformat}{ \vspace*{-12.5mm} – \scshape \chaptername\ \chapterword – }
    \renewcommand*{\chapterheadendvskip}{\vspace*{14.7mm}}
}{}

\ifthenelse{\equal{\papertype}{dina4}}
{
    \renewcommand*{\chapterformat}{ – \scshape \chaptername\ \chapterword – }
}{}

% Aussehen der Seiten (Kapitel- und Textseiten)
\renewcommand*{\partpagestyle}{empty} % Keine Seitenzahl auf der Seite mit \part
\renewcommand*{\chapterpagestyle}{empty} % Keine Seitenzahl auf der Seite mit \chapter
\renewcommand{\chaptermark}[1]{\markboth{#1}{}} % verhindert, dass auf der Kapitelnamenseite davor 'Kapitel 1 -' steht
% Schriftart und -aussehen
\addtokomafont{section}{\normalfont\large\bfseries\centering}
\addtokomafont{subsection}{\normalfont\bfseries\centering}
\setkomafont{pageheadfoot}{\normalfont\scshape}
\pagestyle{scrheadings}
% Wo stehen Seitenzahlen und andere Markierungen auf der Seite
\ohead{\pagemark} % Seitenzahlen oben außen
\ofoot{} % nichts unten außen
\cehead{Harry Potter} % bei geraden Seiten in der Mitte diesen Text schreiben
\cohead{\leftmark} % bei ungeraden Seiten in der Mitte Kapitelname

%\ihead{ihead}
%\chead{chead}
%\ohead{ohead}

%\ifoot{ifoot}
%\cfoot{cfoot}
%\ofoot{ofoot}
